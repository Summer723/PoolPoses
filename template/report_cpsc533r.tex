\documentclass[10pt,twocolumn,letterpaper]{article}

\usepackage{cvpr}
\usepackage{times}
\usepackage{epsfig}
\usepackage{graphicx}
\usepackage{amsmath}
\usepackage{amssymb}

% Include other packages here, before hyperref.

% If you comment hyperref and then uncomment it, you should delete
% egpaper.aux before re-running latex.  (Or just hit 'q' on the first latex
% run, let it finish, and you should be clear).
\usepackage[breaklinks=true,bookmarks=false]{hyperref}

\def\cvprPaperID{Group 1}
%\cvprfinalcopy % *** Uncomment this line for the final submission

\def\httilde{\mbox{\tt\raisebox{-.5ex}{\symbol{126}}}}

\usepackage{color}
% !TEX root = ../top.tex
% !TEX spellcheck = en-US


\definecolor{olive}{RGB}{50,150,50}

\newif\ifdraft
\drafttrue

% project specific, replace!
\newcommand{\netL}{{\it LayerNet}}
\newcommand{\netK}{{\it PiCNet}}
\newcommand{\netU}{{\it KeypointNet}}
\newcommand{\lost}{{\it LostGAN}}

\newcommand{\hA}{{\it ScaleHeuristic}}
\newcommand{\hB}{{\it OverlapHeuristic}}

\newcommand{\soft}{{\it soft-occlusion}}
\newcommand{\lsoft}{{\it local-soft-occlusion}}

\newcommand{\ST}{\mathcal{T}}
\newcommand{\SST}{\mathcal{T}_S}

\newcommand{\R}{\mathbb{R}}
\newcommand{\Seg}{\mathbf{S}} % geometric part
\newcommand{\Latent}{\mathbf{L}}
\newcommand{\LatentG}{\Latent^{\text{3D}}} % geometric part
\newcommand{\LatentA}{\Latent^\text{app}} % appearance part
\newcommand{\LatentBG}{\mB} % appearance part

\newcommand{\dA}{{A}}
\newcommand{\dB}{{B}}

% user specific comments
\newcommand{\HR}[1]{{\color{blue}{\bf helge: #1}}}
\newcommand{\hr}[1]{{\color{blue} #1}}
\newcommand{\XY}[1]{{\color{blue}{\bf xy: #1}}}
\newcommand{\xy}[1]{{\color{blue} #1}}

% general commands
\newcommand{\TODO}[1]{\textcolor{red}{TODO: #1}}
\newcommand{\NEW}[1]{{\color{cyan}{#1}}}
\newcommand{\comment}[1]{}

% sections
\newcommand{\app}{appendix} % supplemental document
\newcommand{\parag}[1]{{\bf{#1}}}

% common math functions
\DeclareMathOperator{\erf}{erf}
\DeclareMathOperator{\smoothstep}{smooth-step}

% vectors and matrix notation
\newcommand{\va}{\mathbf{a}}
\newcommand{\vb}{\mathbf{b}}
%\newcommand{\vc}{\mathbf{c}}
\newcommand{\vd}{\mathbf{d}}
\newcommand{\ve}{\mathbf{e}}
\newcommand{\vf}{\mathbf{f}}
\newcommand{\vg}{\mathbf{g}}
\newcommand{\vh}{\mathbf{h}}
\newcommand{\vi}{\mathbf{i}}
\newcommand{\vj}{\mathbf{j}}
\newcommand{\vk}{\mathbf{k}}
\newcommand{\vl}{\mathbf{l}}
\newcommand{\vm}{\mathbf{m}}
\newcommand{\vn}{\mathbf{n}}
\newcommand{\vo}{\mathbf{o}}
\newcommand{\vp}{\mathbf{p}}
\newcommand{\vq}{\mathbf{q}}
\newcommand{\vr}{\mathbf{r}}
%\newcommand{\vs}{\mathbf{s}}
\newcommand{\vt}{\mathbf{t}}
\newcommand{\vu}{\mathbf{u}}
\newcommand{\vv}{\mathbf{v}}
\newcommand{\vw}{\mathbf{w}}
\newcommand{\vx}{\mathbf{x}}
\newcommand{\vy}{\mathbf{y}}
\newcommand{\vz}{\mathbf{z}}

\newcommand{\mA}{\mathbf{A}}
\newcommand{\mB}{\mathbf{B}}
\newcommand{\mC}{\mathbf{C}}
\newcommand{\mD}{\mathbf{D}}
\newcommand{\mE}{\mathbf{E}}
\newcommand{\mF}{\mathbf{F}}
\newcommand{\mG}{\mathbf{G}}
\newcommand{\mH}{\mathbf{H}}
\newcommand{\mI}{\mathbf{I}}
\newcommand{\mJ}{\mathbf{J}}
\newcommand{\mK}{\mathbf{K}}
\newcommand{\mL}{\mathbf{L}}
\newcommand{\mM}{\mathbf{M}}
\newcommand{\mN}{\mathbf{N}}
\newcommand{\mO}{\mathbf{O}}
\newcommand{\mP}{\mathbf{P}}
\newcommand{\mQ}{\mathbf{Q}}
\newcommand{\mR}{\mathbf{R}}
\newcommand{\mS}{\mathbf{S}}
\newcommand{\mT}{\mathbf{T}}
\newcommand{\mU}{\mathbf{U}}
\newcommand{\mV}{\mathbf{V}}
\newcommand{\mW}{\mathbf{W}}
\newcommand{\mX}{\mathbf{X}}
\newcommand{\mY}{\mathbf{Y}}
\newcommand{\mZ}{\mathbf{Z}}

\newcommand{\cA}{\mathcal A}
\newcommand{\cB}{\mathcal B}
\newcommand{\cC}{\mathcal C}
\newcommand{\cD}{\mathcal D}
\newcommand{\cE}{\mathcal E}
\newcommand{\cF}{\mathcal F}
\newcommand{\cG}{\mathcal G}
\newcommand{\cH}{\mathcal H}
\newcommand{\cI}{\mathcal I}
\newcommand{\cJ}{\mathcal J}
\newcommand{\cK}{\mathcal K}
\newcommand{\cL}{\mathcal L}
\newcommand{\cM}{\mathcal M}
\newcommand{\cN}{\mathcal N}
\newcommand{\cO}{\mathcal O}
\newcommand{\cP}{\mathcal P}
\newcommand{\cQ}{\mathcal Q}
\newcommand{\cR}{\mathcal R}
\newcommand{\cS}{\mathcal S}
\newcommand{\cT}{\mathcal T}
\newcommand{\cU}{\mathcal U}
\newcommand{\cV}{\mathcal V}
\newcommand{\cW}{\mathcal W}
\newcommand{\cX}{\mathcal X}
\newcommand{\cY}{\mathcal Y}
\newcommand{\cZ}{\mathcal Z}




%\usepackage{soul} % strike through

% Pages are numbered in submission mode, and unnumbered in camera-ready
%\ifcvprfinal\pagestyle{empty}\fi
\setcounter{page}{1}

\begin{document}
	

%%%%%%%%% TITLE
\title{Author Guidelines for CPSC 533R}

\author{First Author\\
Institution1\\
Institution1 address\\
{\tt\small firstauthor@i1.org}
% For a paper whose authors are all at the same institution,
% omit the following lines up until the closing ``}''.
% Additional authors and addresses can be added with ``\and'',
% just like the second author.
% To save space, use either the email address or home page, not both
\and
Second Author\\
Institution2\\
First line of institution2 address\\
{\tt\small secondauthor@i2.org}
}

\maketitle

%%%%%%%%% ABSTRACT
\begin{abstract}
   The ABSTRACT should be no longer than half a text column. Follow the following structure (see latex):
   % Sentence 1: CONTEXT - why now?
   % Sentence 2: NEED - why does the reader care?
   % Sentence 3: TASK - what do we do?
   % Sentence 4: OBJECT - what does this document do?
   % Sentence 5: FINDINGS - what did we discover?
   % Sentence 6: CONCLUSIONS - so what?
   % Sentence 7: PERSPECTIVES - what now?
\end{abstract}

%%%%%%%%% BODY TEXT
\section{Author Contribution (include in submission)}
This section is specific to CPSC 533R. Please list the contribution of every team member in terms of development and writing.
It must be balanced for all group members; we assign a single grade for the entire team (unless not justified), see the checklist. 

If this project is closely related to your thesis work, please list your supervisors and explain their level of involvement. In particular, how often did you discuss this project with them and receive feedback?

\subsection{Latex best practices}

We included a defs.tex file that defines useful shortcuts, such as \textbackslash R for $\R$, \textbackslash vp for a vector $\vp$ (bold lower case), and \textbackslash M for a matrix $\mM$ (bold uppercase). We strongly recommend using these to easy typing equations. 

The rest of this template is based on the official CVPR2020 template. Please follow its format.

%%%%%%%%% BODY TEXT
\section{Introduction}

\TODO{Please start from this template for every report milestone submission.} Update all previous sections based on our feedback (e.g., improve the abstract when working on the method section). Only the final report is graded but intermediate submissions are required. Read and follow the suggestions outlined in this document unless you have a good reason. \HR{The tex/defs.tex file defines useful tools for commenting and math formatting, refine these to your initials.} 

%-------------------------------------------------------------------------

\subsection{Paper length}
Reports, excluding the references and author contribution sections,
must be no longer than \hr{six pages in length}. The references section
will not be included in the page count, and there is no limit on the
length of the references section and appendix. 

\subsection{Mathematics}

Use bold lower case letters for vectors and bold upper-case letters for matrices, e.g., $\va$ and $\mA$. It is convenient to use the \textbackslash va and \textbackslash mA commands defined in tex/defs.tex. 

Please number all of your sections and displayed equations.  It is
important for readers to be able to refer to any particular equation.  Just
because you didn't refer to it in the text doesn't mean some future reader
might not need to refer to it.  It is cumbersome to have to use
circumlocutions like ``the equation second from the top of page 3 column
1''.  

Please submit the document in review mode (comment out \textbackslash cvprfinalcopy in this file). Note that the ruler will not be present in the final copy of a paper. Hence, line numbers are not an alternative to equation numbers.  All authors will benefit from reading
Mermin's description of how to write mathematics:
\url{http://www.pamitc.org/documents/mermin.pdf}.

\begin{figure}[t]
\begin{center}
\fbox{\rule{0pt}{2in} \rule{0.9\linewidth}{0pt}}
   %\includegraphics[width=0.8\linewidth]{egfigure.eps}
\end{center}
   \caption{Example of caption.  It is set in Roman so that mathematics
   (always set in Roman: $B \sin A = A \sin B$) may be included without an
   ugly clash.}
\label{fig:long}
\label{fig:onecol}
\end{figure}

\subsection{Miscellaneous}

\noindent
Compare the following:\\
\begin{tabular}{ll}
 \verb'$conf_a$' &  $conf_a$ \\
 \verb'$\mathit{conf}_a$' & $\mathit{conf}_a$\\
\end{tabular}\\
See The TeX book, p165.

The space after \eg, meaning ``for example'', should not be a
sentence-ending space. So \eg is correct, {\em e.g.} is not.  The provided
\verb'\eg' macro takes care of this.

When citing a multi-author paper, you may save space by using ``et alia'',
shortened to ``\etal'' (not ``{\em et.\ al.}'' as ``{\em et}'' is a complete word.)
However, use it only when there are three or more authors.  Thus, the
following is correct: ``
   Frobnication has been trendy lately.
   It was introduced by Alpher~\cite{Alpher02}, and subsequently developed by
   Alpher and Fotheringham-Smythe~\cite{Alpher03}, and Alpher \etal~\cite{Alpher04}.''

This is incorrect: ``... subsequently developed by Alpher \etal~\cite{Alpher03} ...''
because reference~\cite{Alpher03} has just two authors.  If you use the
\verb'\etal' macro provided, then you need not worry about double periods
when used at the end of a sentence as in Alpher \etal.

For this citation style, keep multiple citations in numerical (not
chronological) order, so prefer \cite{Alpher03,Alpher02,Authors14} to
\cite{Alpher02,Alpher03,Authors14}.


\begin{figure*}
\begin{center}
\fbox{\rule{0pt}{2in} \rule{.9\linewidth}{0pt}}
\end{center}
   \caption{Make sure that your figures are all within the page limit and before the references.}
\label{fig:short}
\end{figure*}


%-------------------------------------------------------------------------
\subsection{Footnotes}

Please use footnotes\footnote {This is what a footnote looks like.  It
often distracts the reader from the main flow of the argument.} sparingly.
Indeed, try to avoid footnotes altogether and include necessary peripheral
observations in
the text (within parentheses, if you prefer, as in this sentence).  If you
wish to use a footnote, place it at the bottom of the column on the page on
which it is referenced. Use Times 8-point type, single-spaced.

%-------------------------------------------------------------------------
\subsection{Illustrations, graphs, and photographs}

All graphics should be centered.  Please ensure that any point you wish to
make is resolvable in a printed copy of the paper.  Resize fonts in figures
to match the font in the body text, and choose line widths which render
effectively in print.  Many readers (and reviewers), even of an electronic
copy, will choose to print your paper in order to read it.  You cannot
insist that they do otherwise, and therefore must not assume that they can
zoom in to see tiny details on a graphic.

When placing figures in \LaTeX, it's almost always best to use
\verb+\includegraphics+, and to specify the  figure width as a multiple of
the line width as in the example below
{\small\begin{verbatim}
   \usepackage[dvips]{graphicx} ...
   \includegraphics[width=0.8\linewidth]
                   {myfile.eps}
\end{verbatim}
}


{\small
\bibliographystyle{ieee_fullname}
\bibliography{egbib}
}

\end{document}
